\problemname{DA-Sort}

You recently learned a new way to sort an array of numbers in your algorithms course. The algorithm
sorts an array of numbers by repeatedly performing the \emph{Delete-and-Append} operation. The 
\emph{Delete-and-Append} operation consists of three steps:

\begin{enumerate}
\item Choose an element from the array.
\item Delete the chosen element from the array.
\item Append the chosen element to the end of the array.
\end{enumerate}

Being a curious student, you wonder what is the minimum number of \emph{Delete-and-Append} 
operations required to sort a given array.

\section*{Input}

The first line of input contains a single decimal integer $P$, ($1 \le P \le 100$), which is the number of
data sets that follow. Each data set should be processed identically and independently.

Each data set consists of two or more lines of input. The first line contains the data set number, $K$,
followed by a single space, followed by an integer $N$, ($1 \le N \le 1\,000$), which is the length of the
array to sort. The remaining lines in the dataset contains $N$ positive integers that comprise the array
to be sorted, $10$ values per line, except for the last line which may have less than $10$ values. All the
array elements are no larger than $10^9$. The same value may appear more than once in the array to
be sorted.

\section*{Output}

For each data set there is one line of output. The single output line consists of the data set number,
$K$, followed by a single space followed by an integer which is the minimum number of \emph{Delete-and-Append}
operations required to sort the array.

