\problemname{Which Base is it Anyway?}

Programming languages such as C++ and Java can prefix characters to denote the base of constant
integer values. For example, hexadecimal (base $16$) constants are preceded by the string ``\texttt{0x}''. 
Octal (base $8$) values are preceded by the character ``\texttt{0}'' (zero). Decimal (base $10$) values 
do not have a prefix. For example, all the following represent the same integer constant, albeit 
in different bases.

\begin{center}
\begin{verbatim}
0x1234
011064
 4660
\end{verbatim}
\end{center}

The prefix makes it clear to the compiler what base the value is in. Without the ``\texttt{0x}'' prefix, for
example, it would be impossible for the compiler to determine if 1234 was hexadecimal. 
It could be octal or decimal.

For this problem, you will write a program that interprets a string of decimal digits as if it were an octal
value, a decimal value or a hexadecimal value.

\section*{Input}

The first line of input contains a single decimal integer $P$, ($1 \le P \le 100$), which is the number of
data sets that follow. Each data set should be processed identically and independently.

Each data set consists of a single line of input. It contains the data set number, $K$, followed by a
single space, followed by a string of at most $7$ decimal digits.

\section*{Output}

For each data set there is one line of output. The single output line consists of the data set number,
$K$, followed by a space followed by $3$ space separated decimal integers which are the value of the
input as if it were interpreted to as octal, decimal and hexadecimal respectively. If the input value
cannot be interpreted as an octal value, use the value $0$.

