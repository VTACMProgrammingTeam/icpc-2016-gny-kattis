\problemname{m-ary Partitions}

A partition of an integer $n$ is a set of positive integers which sum to $n$, 
typically written in descending order. For example:

\begin{verbatim}
        10 = 4+3+2+1
\end{verbatim}

A partition is $m$-ary if each term in the partition is a power of $m$. 
For example, the $3$-ary partitions of $9$ are:

\begin{verbatim}
        9
        3+3+3
        3+3+1+1+1
        3+1+1+1+1+1+1
        1+1+1+1+1+1+1+1+1
\end{verbatim}

Write a program to find the number of $m$-ary partitions of an integer $n$.

\section*{Input}

The first line of input contains a single decimal integer $P$, ($1 \le P \le 1\,000$), which is the number of
data sets that follow. Each data set should be processed identically and independently.

Each data set consists of a single line of input. The line contains the data set number, $K$, 
followed by the base of powers, $m$, ($3 \le m \le 100$), followed by a space, followed by the integer,
$n$, ($3 \le n \le 10\,000$), for which the number of $m$-ary partitions is to be found.

\section*{Output}

For each data set there is one line of output. The output line contains the data set number, $K$, a
space, and the number of $m$-ary partitions of $n$. The result should fit in a 32-bit unsigned integer.
